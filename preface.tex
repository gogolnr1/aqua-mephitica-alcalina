\documentclass[a5paper,8pt]{scrbook}
\usepackage[
  top=2cm,bottom=2cm,
  inner=1.75cm,
  outer=1.25cm,
  textwidth=13cm
]{geometry}

\usepackage[utf8]{inputenc}
\usepackage[german]{babel}

\usepackage{palatino}
\usepackage[T1]{fontenc}

\usepackage{graphicx}
\usepackage{hyperref}
\usepackage{hyphenat}
\usepackage{fancyhdr}

\renewcommand{\headrulewidth}{0pt}
\pagestyle{fancyplain} % Makes all pages in the document conform to the custom headers and footers
\fancyhead[L]{}% Empty left header
\fancyhead[C]{\thepage} % Page numbering for center header
\fancyhead[R]{}% Empty right header
\fancyfoot[L]{}% Empty left footer
\fancyfoot[C]{}% Empty center footer
\fancyfoot[R]{}% Empty left footer

\title{Mineralwasser, reich an Hydrogencarbonat\\nach Art von\\Abraham van Stipriaan Luïsçius}
\author{Thorsten van Stipriaan}
\date{September 2021}

\begin{document}

\pagenumbering{gobble}

\maketitle

\newpage

{\centering
{\Large Vorwort}\vfill

\begin{minipage}{4cm}
  \centering
  \hrulefill\\
\end{minipage}

}
\vfill
Abraham Gerardus van Stipriaan Luïsçius%
\footnote{\href{https://web.archive.org/web/20210927050342/https://www.geni.com/people/Abraham-van-Stipriaan-Luïscius/6000000011599769478}
{geni.com/people/Abraham-van-Stipriaan-Luïscius/6000000011599769478}}
konnte am 10. August 1798
seinen nachfolgend behandelten Artikel in der holländischen Zeitung
\emph{Nieuwe allgemeene Konst - en Letter Bode}
abdrucken lassen.
Dieser wurde ein Jahr später
von Dr. August Friedrich Adrian Diel%
\footnote{\href{https://web.archive.org/web/20160828042236/https://de.wikipedia.org/wiki/Adrian_Diel}
{de.wikipedia.org/wiki/Adrian\textunderscore Diel}}
ins Deutsche übersetzt und unter dem Titel
\emph{Art und Weise,
um das laugensalzige Luftsauerwasser
(\texttt{aqua mephitica alcalina})
mit leichter Mühe, und ohne Kosten
vermittelst des Fachinger Mineralwassers zuzubereiten}%
\footnote{\href{https://archive.org/details/b30350360}
{https://archive.org/details/b30350360}}
dem deutschsprachigen Raum in broschierter Form
zur Verfügung gestellt.

Im Folgenden habe ich erwähnte Schrift
in eine leichter lesbare Form übertragen --
dabei Grammatik beibehalten,
Orthografie nach persönlichem Ermessen angepasst
und erweiterte Fußnoten eingesetzt,
um z.B. verwendete Einheiten in metrische Maße umzurechnen
oder, soweit wie möglich, modernere Terminologie anzubieten.
Das Originaldokument von Dr. Friedr. Diel
wird in diesem Jahr genau 222 Jahre alt.

Die Motivation und das Pflichtgefühl dahinter ist
die Hoffnung,
ähnlich die der von A. v. Stipriaan Luïsçius,
dass sich mehr ``Landsleute'' informieren
und ein mit wesentlichen nicht-organischen Salzen %essenziellen Spurenelementen
angereichertes Wasser
%für einen  gesundheitlichen Zustand
in die alltägliche Anwendung bringen können.
Beim Lesen habe ich ein einprägsames Gefühl für den Wert
eines solchen Wassers bekommen,
insbesondere durch den Charakter --
sowohl den der damaligen Zeit, als auch den der Autoren.

Falls der Leser
stellenweise an alten Formulierungsarten
oder zu technischen Inhalten anecken sollte,
ermutige ich diesen,
solche Stellen zu überspringen und dort fortzusetzen,
wo wieder mehr Lesefluss möglich ist.
Es lohnt sich meiner Meinung nach,
dadurch ein Gefühl für die diskrete Dringlichkeit zu bekommen,
welche heute mindestens so aktuell ist, wie damals.
Die praktische Anwendung ist heutzutage um einiges einfacher,
welche ich in nachfolgenden Kapiteln beschreiben werde.

Wer sich berufen fühlt, dieser Schrift etwas beizutragen,
ist herzlich willkommen, dieses auf Github
\footnote{}
unter der Common Creative Lizenz
\footnote{}
zu tun.\\

\pagenumbering{Roman}
\setcounter{page}{3}

Nachfolgend an die Transkription folgen:
\begin{itemize}
\item Analyse der Vorgehensweise von A. v. Stipriaan Luïsçius
in Grammatik, Begriffen und Werten der heutigen Zeit.
\item Anmerkungen und Vergleiche zum heutzutage käuflichen
sog. ``Staatl. Fachingen Wassers'',
um eine Perspektive aufzuzeigen,
warum es dennoch sinnvoll sein kann,
ein solches Wasser selber herzustellen
oder alternative Angebote von Elektrolyt-Wasser zum Kauf anzubieten
\item zeitgemäße Wege und Ansätze, ein qualitativ hochwertiges Wasser selber herzustellen
\end{itemize}

%Um letzterem eine echte und faire Chance zu geben,
%sehe ich mittelfristig einem Crowd-Funding Projekt entgegen,
%um solch ein heilsames Wasser für die Masse verfügbar zu machen,
%da ich von dessen Wirksamkeit ebenfalls von ganzem Herzen überzeugt bin.

Es war eine kleine Reise,
erst das Dokument zu lesen,
ohne im ersten Anlauf den Inhalt im Detail verstanden zu haben --
aber wohl erkennen konnte,
dass hier ein kleiner Schatz verborgen liegt;
dann das Dokument abzutippen
und in eine neues Format zu übertragen,
dabei ins tiefere Verständnis zu kommen,
etwas über Chemie, Physiologie und Geschichte zu lernen;
dann im Detail zu analysieren, umzurechnen,
den Aufbau und die Chemie,
alles nach bestem Wissen und Gewissen nachzuvollziehen
\emph{und} -- ganz am Ende -- den tatsächlichen Schatz zu heben.
Denn als ich das Beschriebene begriff,
war der ``Zauber'' kurz vorbei
und dachte: \emph{wie einfach!}
Und mit Hinblick auf moderne und alternative Forschungen
konnte ich weitere Zusammenhänge erkennen und herstellen,
wie bahnbrechend das Ganze selbst für unsere heutige Zeit ist,
vorausgesetzt es würden sich genügend Menschen
solches Wissen zu Eigen und Nutzen machen.

\end{document}
